\documentclass{beamer}

\usepackage{beamerthemesplit} % // Activate for custom appearance
\usepackage{tikz} % For graphics
\usepackage{listings}
\usepackage{color}
\usepackage{upquote}


\definecolor{lightgray}{rgb}{.9,.9,.9}
\definecolor{darkgray}{rgb}{.4,.4,.4}
\definecolor{darkblue}{rgb}{0,0,0.4}
\definecolor{darkgreen}{rgb}{0,0.4,0}
\definecolor{purple}{rgb}{0.65, 0.12, 0.82}
\definecolor{editorGray}{rgb}{0.95, 0.95, 0.95}
\definecolor{editorOcher}{rgb}{1, 0.5, 0} % #FF7F00 -> rgb(239, 169, 0)
\definecolor{editorGreen}{rgb}{0, 0.5, 0} % #007C00 -> rgb(0, 124, 0)

\lstdefinelanguage{JavaScript}{
  keywords={typeof, new, true, false, catch, function, return, null, catch, switch, var, const, class, import, export, from, if, in, while, do, else, case, break},
  otherkeywords={% Operators
    (, ), (), =>, =, >, <, ==
  },
  keywordstyle=\color{darkgreen}\bfseries,
  ndkeywords={class, export, boolean, throw, implements, import, this},
  ndkeywordstyle=\color{darkblue}\bfseries,
  numberstyle=\color{red},
  identifierstyle=\color{black},
  sensitive=false,
  comment=[l]{//},
  morecomment=[s]{/*}{*/},
  commentstyle=\color{darkgray}\ttfamily,
  stringstyle=\color{purple}\ttfamily,
  morestring=[b]',
  morestring=[b]"
}


\lstset{
  language=JavaScript,
  backgroundcolor=\color{lightgray},
  extendedchars=true,
  basicstyle=\footnotesize\ttfamily,
  showstringspaces=false,
  showspaces=false,
  numbers=left,
  numberstyle=\footnotesize,
  numbersep=9pt,
  tabsize=2,
  breaklines=true,
  showtabs=false,
  captionpos=b
}


\title{React Workshop}
\author{Csaba Palankai}
\subtitle{A backend python developer adventures on the frontend side}
\date{\today}

\begin{document}

\frame{\titlepage}

\section{Introduction}

\subsection{What we are going to do}


\frame
{
  \frametitle{Disclaimer}
  If you follow this workshop you will be able to start
  a React application development. 
  \pause
  Including write tests from the beginning.
  \pause
  Building assets together, using latest javascript features.
  \pause
  Finally, render the application server side and client side as well.
}

\frame
{
  \frametitle{Disclaimer}
  I'm new to React, very new.
  \pause
  When I wrote down this line I had no idea how could I
  complete this challenge of building this application with every mentioned feature.
  \pause
  
  
  I'm not a frontend developer. I have some javascript and CSS experience, but very rusty.
  \pause
  I have no design skills
  \pause
  (at all).
  \pause
  You will see.
  \pause
  However I was able to learn how to build websites with React.
  \pause
}

\frame
{
  \frametitle{Features of this workshop}

  \begin{itemize}
  \item<1-> Initialise docker working environment
  \item<2-> Create a React project project
  \item<3-> Make a few React components
  \item<4-> Write tests agains our components
  \item<5-> Build multiple pages using React Router
  \item<6-> Generate our HTML server side
  \item<7-> Fill up the site with data
  \item<8-> Prettify with SCSS
  \item<9-> Refactor a bit, make it production ready
  \end{itemize}
}

\subsection{Why I am doing this}

\frame
{
  \frametitle{Disclaimer}
  
  I'm not a frontend developer. I have some javascript and CSS experience, but very rusty.
  \pause
  I have no design skills
  \pause
  (at all).
  \pause
  You will see.
  \pause
  However I was able to learn how to build websites with React.
  \pause
}


\frame
{
  \frametitle{My motivation}
  I wanted to build some admin interface, visualise some data.
  I had a few options:

  \begin{itemize}
  \item<1-> Build every page server side
  \item<2-> A frontend javascript framework
  \end{itemize}
  \pause
  \pause
  
  I've started to read about the options. I found React offer very attractive.
  
  \begin{itemize}
  \item<3-> Simple syntax and client side logic
  \item<4-> Big community, plenty of 3rd party addons
  \item<5-> Easy to test (although not in every case)
  \item<6-> Can be rendered (almost) effortless server side as well
  
  \end{itemize}
}



\section{What is react}

\begin{frame}
  \frametitle{A component definition}
  React is a Javascript framework which only supports
  component rendering.
  \pause
  No data handling (AJAX)

\end{frame}


\begin{frame}[fragile]
  \frametitle{A component definition}
   \begin{verbatim}
import React from 'react';
   
class MyComponent extends React.Component {
  render() {
    return (
      <div>
        <h1>Header of the page</h1>
        <p>Some text</p>
      </div>
    )
  }
}
   \end{verbatim}
\end{frame}

\begin{frame}[fragile]
  \frametitle{A component definition shorter}
   \begin{verbatim}
import React from 'react';
   
function MyComponent() {
  return (
    <div>
      <h1>Header of the page</h1>
      <p>Some text</p>
    </div>
  );
}
   \end{verbatim}
\end{frame}


\begin{frame}[fragile]
  \frametitle{A component definition even shorter}
   \begin{verbatim}
import React from 'react';
   
const MyComponent = () => {
  return (
    <div>
      <h1>Header of the page</h1>
      <p>Some text</p>
    </div>
  );
}
   \end{verbatim}
\end{frame}


\begin{frame}[fragile]
  \frametitle{A compact component definition}
   \begin{verbatim}
import React from 'react';
   
const MyComponent = () => (
  <div>
    <h1>Header of the page</h1>
    <p>Some text</p>
  </div>
);
   \end{verbatim}
\end{frame}


\begin{frame}
  \frametitle{New page}
  Some text
\end{frame}

\begin{frame}
  \frametitle{New page}
  Some text
\end{frame}

\begin{frame}
\begin{figure}[h!]
  \begin{center}
    \begin{tikzpicture}
      \draw [red,dashed] (-2.5,2.5) rectangle (-1.5,1.5) node [black,below] {Start}; % Draws a rectangle
      \draw [thick] (-2,2) % Draws a line
      to [out=10,in=190] (2,2)
      to [out=10,in=90] (6,0) 
      to [out=-90,in=30] (-2,-2);    
      \draw [fill] (5,0.1) rectangle (7,-0.1) node [black,right] {Obstacle}; % Draws another rectangle
      \draw [red,fill] (-2,-2) circle [radius=0.2] node [black,below=4] {Point of interest}; % Draws a circle
    \end{tikzpicture}
    \caption{Example graphic made with tikz.}
  \end{center}
\end{figure}
\end{frame}


\begin{frame}[fragile]
\begin{lstlisting}
import React from 'react';

// Some comment
/* some more comment */   
const MyComponent = () => (
  <div>
    <h1>Header of the page</h1>
    <p>Some text</p>
  </div>
);
\end{lstlisting}
\end{frame}


\end{document}

